@node machine,,syscalls,Top
@chapter NEC V70 Specific Functions

The NEC V70 has machine instructions for fast IEEE floating-point
arithmetic, including operations normally provided by the library.  

When you use the @file{/usr/include/fastmath.h} header file, the
names of several library math functions are redefined to call the
@code{fastmath} routine (using the corresponding V70 machine instructions)
whenever possible.

For example,
@example

#include <fastmath.h>

double sqsin(x)
double x;
@{
  return sin(x*x);
@}

@end example
expands into the code
@example

@dots{}
double sqsin(x)
double x;
@{
  return fast_sin(x*x);
@}

@end example

The library has an entry @code{fast_sin} which uses the machine
instruction @code{fsin.l} to perform the operation.  Note that the
built-in instructions cannot set @code{errno}
in the same way that the C coded functions do.  Refer to a V70
instruction manual to see how errors are generated and handled.

Also, the library provides true @code{float} entry points.  The
@code{fast_sinf} entry point really performs a @code{fsin.s}
operation.  Because of this, the instructions are only useful when
using code compiled with an ANSI C compiler.  The prototypes
and definitions for the floating-point versions of the math library
routines are only defined if compiling a module with an ANSI C
compiler.

@page
@section Entry points 
The functions provided by @file{fastmath.h} are
@example

 double fast_sin(double);	/*	fsin.l */
 double fast_cos(double);	/*	fcos.l */
 double fast_tan(double);	/*	ftan.l */
 double fast_asin(double);	/*	fasin.l */
 double fast_acos(double);	/*	facos.l */
 double fast_atan(double);	/*	fatan.l */
 double fast_sinh(double);	/*	fsinh.l */
 double fast_cosh(double);	/*	fcosh.l */
 double fast_tanh(double);	/*	ftanh.l */
 double fast_asinh(double);	/*	fasinh.l */
 double fast_acosh(double);	/*	facosh.l */
 double fast_atanh(double);	/*	fatanh.l */
 double fast_fabs(double);	/*	fabs.l */
 double fast_sqrt(double);	/*	fsqrt.l */
 double fast_exp2(double);	/*	fexp2.l */
 double fast_exp10(double);	/*	fexp10.l */
 double fast_expe(double);	/*	fexpe.l */
 double fast_log10(double);	/*	flog10.l */
 double fast_log2(double);	/*	flog2.l */
 double fast_loge(double);	/*	floge.l */

 float fast_sinf(float);	/*	fsin.s */
 float fast_cosf(float);	/*	fcos.s */
 float fast_tanf(float);	/*	ftan.s */
 float fast_asinf(float);	/*	fasin.s */
 float fast_acosf(float);	/*	facos.s */
 float fast_atanf(float);	/*	fatan.s */
 float fast_sinhf(float);	/*	fsinh.s */
 float fast_coshf(float);	/*	fcosh.s */
 float fast_tanhf(float);	/*	ftanh.s */
 float fast_asinhf(float);	/*	fasinh.s */
 float fast_acoshf(float);	/*	facosh.s */
 float fast_atanhf(float);	/*	fatanh.s */
 float fast_fabsf(float);	/*	fabs.s */
 float fast_sqrtf(float);	/*	fsqrt.s */
 float fast_exp2f(float);	/*	fexp2.s */
 float fast_exp10f(float);	/*	fexp10.s */
 float fast_expef(float);	/*	fexpe.s */
 float fast_log10f(float);	/*	flog10.s */
 float fast_log2f(float);	/*	flog2.s */
 float fast_logef(float);	/*	floge.s */

@end example


